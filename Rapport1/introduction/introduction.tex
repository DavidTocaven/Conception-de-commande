\chapter*{Introduction}
\addcontentsline{toc}{chapter}{Introduction}
\label{chap:Intro}

Dans cet UE, nous devons réaliser la commande d’un système temps réel du prototypage à l'implémentation sur un microcontrôleur : nous devons asservir un banc de moteurs à courant continu à l'aide d'un micro-contrôleur C167. 

Ce premier rapport contient toute la partie théorique. Celle-ci est décomposée dans les trois premiers chapitres et un dernier chapitre qui détaille la suite des étapes à réaliser.

Le \emph{chapitre \ref{chap:modelisation} : Modélisation} contient l'étude physique qui nous a été donné en cours, le modèle le plus précis, non linéaire et variant, les différentes simplifications de celui-ci. 

Ensuite, dans le \emph{chapitre \ref{chap:analyse} : Analyse}, nous avons effectué une analyse de nos différents modèles afin de maîtriser l’impact de nos simplifications, étudier les performances de notre systèmes et définir celles souhaitées. Nous avons réalisé cela grâce, autant que nous avons pu, a une approche théorique et grâce à des simulations.

Le \emph{chapitre \ref{chap:commande} : Synthèse de commande}, qui est le dernier chapitre de la partie théorique, contient la conception de l'asservissement et l'étude des performances de celle-ci sur les différents modèles de notre système.
 
Dans le \emph{chapitre \ref{chap:suite} : Planification de la suite de l'asservissement }, nous détaillons comment nous allons testé la validité de nos modèles par rapport au modèle physique et les différentes étapes de la mise en \oe uvre sur micro-contrôleur. Ce chapitre nous permettra d'organiser au mieux notre démarche afin que la commande implémentée respecte bien les contraintes temps réel, garantisse la stabilité et les performances attendues tout en étant adaptée au support d'implémentation et au moteur asservi.

 % from sillabus :  les étudiants apprendront à réaliser la commande d’un système temps réel de bout en bout, du prototypage à la mise en œuvre sur un calculateur numérique (type microcontrôleur). Pour cela, ils apprendront à tenir compte des contraintes matérielles de la chaîne de contrôle-commande (i.e. du calculateur au procédé) pour effectuer le bon choix des différentes interfaces et de la meilleure architecture logicielle.