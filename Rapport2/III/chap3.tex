\chapter{Synthèse de commande}
\label{chap:commande}
\section{Commande du système d'ordre 2}
\subsection{Rédaction du cahier des charges et démarche de réponse}
Après l'étude que nous venons de réaliser sur notre système, nous allons ici exprimer les attentes que doit réaliser la commande que nous allons implémenter. Nous souhaitons avoir : \begin{itemize} 
\item Erreur de position nulle.
\item Pas d'oscillations
\item Temps de réponse inférieur à 1 seconde.
\end{itemize}
La commande de notre système doit permettre d'asservir le système en vitesse, par rapport à une consigne. Pour respecter, nous allons réaliser un placement de valeurs propres par retour d'état.



\subsection{Calcul du retour d'état}
\label{sub:Calcul du retour etat}
Pour garantir les performances dynamiques souhaitées, nous allons appliquer un placement de pôles par retour d'état. Ce réajustement des valeurs propres de la matrice dynamique du système nous permettra de répondre aux attentes du cahier des charges si le choix de celles-ci est correct. De même, nous devons choisir des valeurs propres qui ne sont pas être trop éloignées de celles du procédé, pour ne pas être trop exigeant avec la commande et le système. Nous avons avec ces spécifications choisi les valeur propres suivantes :
\begin{equation}
\label{equation:valeurPropres}
\lambda = \begin{pmatrix}
-4 &-5
\end{pmatrix}
\end{equation}
La commande par retour d'état est une méthode d'asservissement qui permet d'effectuer un placement de pôles. Cette correction se réalise grâce à un gain $K$ qui va corriger l'état $x$ d'un système. Cette loi de commande s'écrit : 
\begin{equation}
u(t) = Ny_{ref}-Kx(t)
\end{equation} avec $N$ un gain de pré-compensation, $y_{ref}$ la vitesse de référence, $x(t)$ la sortie mesuré du système et $K$ le gain de retour. Si l'on applique cette loi à notre système en espace d'état, on obtient : 
\begin{align*}
\left\lbrace
\begin{aligned}
&\dot{X}(t) = (A-BK)X(t) + BNy_{ref}\\
&Y = CX(t)
\end{aligned}
\right.
\end{align*}
Ainsi la nouvelle dynamique du système est donnée par la matrice $A' = (A-BK)$ et doit admettre les valeurs propres que nous désirons appliquer à notre système. $A$ et $B$ étant des paramètres du système, nous allons utiliser le gain $K$ pour répondre à ce problème. Avec la fonction $place$ de MATLAB, nous sommes capable de concevoir ce vecteur $K$. 


\subsection{Observateur ordre plein sur modèle d'ordre 2\label{sub:constructionObeservateur}}
Pour pouvoir réaliser notre commande par retour d'état, nous devons tous d'abord reconstruire l'ensemble des états du système dont nous n'avons pas accès. Dans notre cas, nous disposons d'une mesure de la position du moteur $\theta$ avec $V_s$ mais aucune mesure nécessaire pour au minimum reconstruire cet état.


Nous préférons reconstruire les deux états $\Omega$ et $\theta$ à partir de $V_s$ et de l'entrée du modèle d'ordre 2 pour simplifier les calculs nécessaire à sa construction. Les équations de l'observateur sont présentées par :
\begin{align*}
\left\lbrace
\begin{aligned}
&\dot z (t) = Fz(t) + Gy(t) + Hu(t)\\
&\hat x (t)= Mz(t) + Ny(t)\\
&\epsilon (t)= x-\hat{x}
\end{aligned}
\right.
\end{align*}
 où $x$ représente l'état du système, $\hat{x}$ l'état du système reconstruit et $\epsilon$ l'erreur d'estimation à un temps $t$. Nous souhaitons contrôler la dynamique de ce paramètres pour pouvoir estimer correctement notre système. Pour cela, nous nous interressons à : 
\begin{eqnarray}
&&\dot{\epsilon} = \dot{x} - \dot{\hat{x}}\\
\label{equ:obs2}&\Leftrightarrow & \dot{\epsilon} = Ax +Bu - F\hat{x} - Gy - Hu \text{   en considérant }M=1 \text{ et } N = 0\\
& \Leftrightarrow & \dot{\epsilon} = Ax - F\hat{x} - GCx + u(B-H)\\
& \Leftrightarrow & \dot{\epsilon} = (A-GC-F)x + F\epsilon + u(B-H)
\end{eqnarray}
Il vient alors $F = A-GC$ et $B=H$ pour obtenir $\dot{\epsilon} = F\epsilon$. Ainsi l'erreur d'estimation est autonome et ne dépend pas des entrées et sorties du système, et il vient $\epsilon(t) = e^{Ft}\epsilon(0)$. Les valeurs propres de $F$ vont ainsi déterminer la dynamique de l'erreur d'estimation, nous choisissons de les faire dépendre des valeurs propres désirées dans la partie \ref{sub:Calcul du retour etat} en les multipliant par 3, pour une convergence encore plus rapide. 
\begin{equation}\label{eqn:vpObservateur}
v_{p_{obs}} = 3\times \lambda
\end{equation}

\subsection{Construction de l'asservissement}\label{sub:constructionAsservissement}
Maintenant que l'observateur et le gain du retour d'état sont construits, nous allons les réunir pour former un unique modèle de commande. Pour ce modèle du système bouclé, nous prenons comme vecteur d'état $X=\begin{pmatrix}
X&\epsilon
\end{pmatrix}^T$ et nous obtenons par construction :
\begin{align*}
\label{eqn:systemeEE2_bf}
\left\lbrace
\begin{aligned}
& \dot{X} = \begin{pmatrix}
(A-BK) & -BK\\
0& F
\end{pmatrix}X+\begin{pmatrix}
BN\\0
\end{pmatrix}y_{ref} \\
& y(t) = \begin{pmatrix}
C & 0
\end{pmatrix}X
\end{aligned}
\right.
\end{align*} 

Le gain de pré-compensation $N$ est à partir gain statique du transfert entre $Vg$ et $y_{ref}$ noté $G_{Vg/y_{ref}}$ pour permettre une erreur de position nulle. Il est : 
\begin{equation}
N = \frac{1}{G_{Vg/y_{ref}}}
\end{equation}


Cependant, pour permettre un asservissement du système en vitesse $\omega$, nous devons modifier légèrement le résultat obtenu. La description telle qu'elle est donnée de ce système bouclé va asservir le système en position, à cause du changement de dynamiques de la position $\theta$ avec le retour d'état. Pour prévenir à cette mauvaise commande, nous avons annulé le retour de la mesure de position dans le système bouclé. 


A partir de ce modèle, nous allons être capable d'éprouver la commande obtenu de manière théorique et avec une simulation SIMULINK.

\section{Validation de la commande}
\subsection{Validation théorique}
Pour commencer, nous allons identifier les valeurs propres de notre système \ref{eqn:systemeEE2_bf} afin de de connaitre les effets de l'observateur sur le retour d'état, même si a priori celui ci est autonome. Nous obtenons les $\lambda_i$ valeurs propres suivantes : $-5$, $-4$, $-15$ et $-12$. Nous remarquons que les $\lambda_i$ n'ont pas été modifiées : nous avons les valeurs propres désirées par la commande et nous avons celles désirées pour l'erreur d'estimation de l'observateur. 

\subsection{Simulation SIMULINK}
Pour compléter la validation de la commande, nous avons crée un prototype avec SIMULINK pour pouvoir simuler une réponse temporelle de la commande de notre système. La description du fichier se trouve en annexe \ref{Annex:SIMULINK_RE}. Nous observons sur les figures suivantes la réponse temporelle à une référence $y_{ref} = 1$ : 
\begin{figure}[!ht]
\begin{center}
\includegraphics[width = \textwidth]{./III/figure/stepEE2bf.png}
\caption{Réponse du système asservi}
\end{center}
\end{figure}
La réponse du système $Vs$ simulé correspond aux changement de dynamique que nous avons espéré. De plus, après une observation de la commande $u(t)$ appliquée sur le système, nous voyons que la tension maximale envoyée est borné dans l'ordre de grandeur des valeurs admises par le système physique. 


\subsection{Analyse boucle fermé du modèle d'ordre 3}
Notre commande respecte le cahier des charges, nous validons ainsi la commande pour notre modèle d'ordre 2. Nous allons maintenant valider cette même commande sur des modèles supérieur et plus complexes pour compléter sa validation. En effet, notre modèle a été très simplifié pour permettre la création du commande facilement. Si nous arrivons a prouvé que cette commande marche sur des systèmes plus complexes, alors nous aurons prouvé que les dynamiques que nous avons simplifiés ne vont pas perturbé notre commande.

\section{Validation sur les modèles d'ordre supérieur}
\subsection{Changement de base du modèle d'ordre 3}
Dans un premier temps, nous regarderons les effets de l'implantation de l'observateur d'ordre 2 sur les modèles d'ordre supérieur et plus précisément l'erreur d'estimation et le placement des valeurs propres, puis nous simulerons la commande obtenu pour observé les modifications des réponses.

Pour intégrer l'observateur dans un espace d'état d'ordre supérieur, nous avons du réorganiser les états du modèle, ici le modèle d'ordre 3, de façon à ce que les états reconstruits avec l'observateur soient en haut, $x_1$, et les non reconstruits en bas $x_2$.

\noindent\textbullet\hspace{2mm} États reconstruits : $\Omega_m$ et $ \Theta_m$.

\noindent\textbullet\hspace{2mm} États non reconstruits : $i_1$

On obtient le vecteur d'état suivant : 
\begin{equation}
\overline{X} = \begin{bmatrix}
\theta\\
\omega\\
i_1\\
\end{bmatrix} = \begin{pmatrix}
x_1\\x_2
\end{pmatrix}
\end{equation}
donc $x_1 = \begin{bmatrix}
\theta\\
\omega\\ \end{bmatrix}$ et $x_2 = i_1$. Pour passer de $X$ à $\overline{X}$, il faut utiliser une matrice de passage $P_X$ définit par :
 \begin{eqnarray}
 P_X &/&  \overline{X} =P_X \cdot X \\
 P_X &=&\begin{bmatrix}
 0 & 0 & 1 \\
 1 & 0 & 0 \\
 0 & 1 & 0 \\
\end{bmatrix}  \\
 \end{eqnarray}
Maintenant, nous devons calculer $\overline{A}$, $\overline{B}$, $\overline{C}$, $\overline{D}$ pour le nouvel espace d'état lié à $\overline{X}$.
\begin{equation}%
	\left\lbrace%
	\begin{matrix}
		\overline{A} &=& P^{-1} A P \\%
		\overline{B} &=& P^{-1} B \\%
		\overline{C} &=& C P%	
	\end{matrix}
\right.%
\end{equation}

\subsection{Analyse du changement de base et de l'erreur de reconstruction}
Maintenant que nous avons replacé le modèle d'état, nous pouvons commencer l'analyse de celui ci. Avec les résultats obtenus pendant les séances de TD et de cours, nous avons obtenu une relation de la dynamique de l'erreur \begin{equation}\label{eqn:erreurRecompoOrdreSup}
\dot{\epsilon} = F\epsilon - A_2x_2
\end{equation}
avec $A_2$ la partie de la matrice $A$ qui lie l'état $\dot{x_1}$ à $x_2$ dans la nouvelle base. Cet élément modifie les résultats que nous avons obtenu pour le modèle d'ordre 2 en indiquant que l'évolution de l'erreur de reconstruction des états n'est plus en fonction uniquement d'elle même, mais aussi des états qui n'ont pas été reconstruits. Cette nouvelle donné va perturbé la convergence de la reconstruction de l'erreur, nous n'arriverons pas à reconstruire l'ensemble des états. 


Nous allons étudier les nouvelles valeurs propres du système bouclé pour vérifier si celles si correspondent toujours au cahier des charges, en modélisant un espace d'état qui admet comme vecteur d'état $x=\begin{pmatrix}
\overline{X}&\epsilon \end{pmatrix}^T$. Nous obtenons : $-7748$, $-20.61$, $ -3.832 \pm 6.931$ et $-2.784$, la stabilité asymptotique en boucle fermé est toujours respectée.



Pour continuer l'analyse de cette commande, nous obtenons à partir de tracé \emph{MATLAB} les réponses temporelles du système en boucle fermé avec les caractéristiques suivantes : \begin{itemize}
\item Le gain statique n'est plus respecté. Nous commençons à voir apparaitre une erreur de position en régime statique. Cette erreur est due à l'erreur de reconstruction de 
\item le temps de réponse reste dans la clause du cahier des charges et pas d'oscillations
\end{itemize} 


Pour finir l'analyse de la boucle fermé, nous allons étudier le transfert de $V_g(t)$ avec la consigne et le comparer avec celui du modèle d'ordre 2. Nous obtenons le diagramme de Bode en figure \ref{fig:transfertX/consigne}.
\begin{figure}[!ht]
\centering
\includegraphics[width = .5\textwidth]{./III/figure/bode_Tpert_EE1.png}
\caption{Transfert de la boucle fermé, modèle linéaire d'ordre 3\label{fig:transfertX/consigne}}
\end{figure}


Le résultat que nous obtenons confirmes la stabilité asymptotique établie précédemment, cependant les différences notables avec le transfert du modèle d'ordre 2 en basse fréquence explique pourquoi nous n'avons pas pu respecter exactement le cahier des charges. 

\subsection{Changement de base du modèle d'ordre 4 et intégration de la commande\label{sub:observateurSurModelO4}}
De la même manière que pour le modèle d'ordre 3, nous effectuons un changement de base avec la matrice de passage : \begin{equation}
P = \begin{pmatrix}
0 &0& 1& 0\\ 0& 0& 0& 1\\1& 0& 0& 0\\ 0& 1& 0& 0
\end{pmatrix}
\end{equation}
Nous obtenons ainsi comme dans la section précédente, un modèle d'ordre 4 qui a comme vecteur d'état : 
\begin{equation}
\overline{X} = \begin{pmatrix}
\theta & \omega & i_1& i_2
\end{pmatrix}^T
\end{equation}

Les performances du système bouclé ont les mêmes problèmes que pour le modèle d'ordre 3. Les reconstructions des états ne convergent pas vers 0, comme nous l'avions vu précédemment, ce qui perturbe l'asservissement en vitesse.

De plus, les approximations de modélisation ont montré dans l'analyse des modèles une erreur statique que nous ne pouvons pas corriger avec une commande construite sur le modèle d'ordre 2. Nous avons toutefois voulu analyser le transfert entre $Vg$ et la consigne et obtenons :
\begin{figure}[ht]
\centering
\includegraphics[width = .5\textwidth]{./III/figure/bode_Tpert_EE0.png}
\caption{Transfert de la boucle fermé, modèle linéaire d'ordre 4}
\end{figure}
Nous avons de plus amples écart en basse fréquence avec un dépassement encore plus grand que pour le modèle d'ordre 3. Ce dépassement joue beaucoup sur les oscillations du système en boucle fermé. Le cahier des charges n'est encore une fois pas respecté pour ce modèle mais admet un temps de réponse proche de celui du modèle d'ordre 2.

En sachant ceci, nous savons que les dynamiques qui ont été simplifiés ne sont pas l'origine des écarts que nous venons d'exposer, il s'agit de la dynamique de l'observateur qui n'est pas optimale. 

\subsection{Étude de la réponse entre l'erreur de reconstruction par rapport à une commande}\label{sub:etudeTransfertErreurCommande}
Avec les changements de base calculé précédemment, nous avons souhaité analysé les transferts entre l'erreur de reconstruction $\epsilon$ et la commande $u$. Pour cela, nous avons utilisé une réponse temporelle affichée en figure \ref{fig:repTemporelle_erreurReconstruction}.  
\begin{figure}[!ht]
\centering
\includegraphics[width= .9\textwidth]{./III/figure/reponses_temp-erreurDeReconstruction.pdf}
\caption{\label{fig:repTemporelle_erreurReconstruction}Réponses de l'erreur de reconstruction de l'état $\omega$(vitesse) des modèle d'ordre 2 à 4}
\end{figure}Dans cette figure, nous voyons que l'erreur pour le modèle $EE2$ d'ordre 2, l'erreur converge très rapidement vers 0. Pour les autres modèles, l'erreur converge aussi, mais pas vers 0. Ce décalage du régime statique de $\epsilon$ ne nous permettra pas de reconstruire parfaitement l'état de la vitesse du moteur, mais uniquement de nous en approcher. 


\section{Simulation $SIMULINK$ des modèles linéaires}
Maintenant que l'observateur est prêt à être implémenter sur les modèles d'ordre 3 et 4, nous allons pouvoir commencer à exécuter des simulations des commandes de ces deux modèles. Deux choix s'offrent à nous, nous pouvons utilisé le changement de base et donc un système en espace d'état de notre système bouclé. Dans ce cas là, nous aurions accès à la vitesse asservis en fonction d'une référence. Le deuxième choix et celui qui nous semble le plus judicieux, consiste à brancher chaque élément sur un schéma $SIMULINK$. Nous avons ainsi accès à tous les échanges de signaux entre les blocs. 


Les schémas blocs que nous avons simulés sont disponibles en annexe \ref{Annex:SIMULINK_RE}. La réponse temporelle de $V_g(t)$ se trouve, pour les modèles linéaires d'ordre 2 à 3, dans la figure \ref{fig:reponseModelesLineairesSIMULINK}, ainsi que la commande qui est associé à cette réponse, en figure \ref{fig:reponseModelesLineairesSIMULINK}.\begin{figure}[!ht]
\centering
\includegraphics[width = .7\textwidth]{./III/figure/res_SIMULINK.png}
\caption{Réponses temporelles des boucles fermées sous $SIMULINK$\label{fig:reponseModelesLineairesSIMULINK}}
\end{figure}


\begin{figure}[!ht]
\centering
\includegraphics[width = .7\textwidth]{./III/figure/commande-u-ML_ordre2-3-4.pdf}
\caption{Commandes $u(t)$ associées aux résultats de la figure \ref{fig:reponseModelesLineairesSIMULINK}\label{fig:commandesML-1V}}
\end{figure}


Sur ces figures, pour le modèle linéaire d'ordre 4, nous avons obtenu le bon temps de réponse et pas d'oscillations. L'erreur du régime permanent est de $2\%$.
Ces résultats nous permettent de regarder une simulation des signaux de $Vg(t)$. D'après ces résultats, la commande que nous souhaitons appliquer respecte le cahier des charges en termes d'oscillations de la réponse et de temps de réponse mais pas pour le gain statique. Cependant, nous ne pouvons en tirer aucune conclusion , il reste encore d'autres modèles plus complexes sur lequel notre commande doit être tester.

\section{Conclusion}
Nous avons maintenant terminé les exemples de simulations sur les modèles linéaires du moteur que nous avons calculé. A partir de cet instant, nous avons deux possibilités, nous pouvons tester directement le retour d'état sur le moteur, sans vraiment connaitre les conséquences du passage de notre modèle linéaire au procédé réel du moteur, soit nous analysons d'abord la loi de commande créer ici, sur un modèle non linéaire. Nous aborderons cette dernère proposition dans le chapitre prochain.


