\chapter{Bilan }
Nous venons de vous présenter l'ensemble de nos travaux sur la conception de commande temps réel avec : la modélisation du procédé et son analyse, une synthèse de commande avec sa validation ainsi que sa discrétisation, en terminant par son implémentation sur un code en langage C. 


Pour chacune de ces parties, nos compétences acquises au cours de notre cursus ont été utiles et nous avons relevé un grand intérêt dans la mise en relation entre ces différentes connaissances. Nous avons pu faire un lien entre l'automatique linéaire et l'implémentation sur micro-contrôleur, en passant par la discrétisation et l'analyse de prototype avec $MATLAB$. Nous avons vu aussi l'importance lié à une validation des résultats étape après étape, pour ainsi effectuer une mise en œuvre un peu plus réfléchie que durant nos années précédente.

Cependant, nous avons rencontré quelques problèmes qui sont, il nous semble, bon de faire ressortir pour une évolution personnelles. Nous avons été en difficulté quand à la sélection des points précis qu'il fallait aborder. En effet, nous ne pouvions pas utiliser l'intégralité des compétences, un tri a été nécessaire et nous avons été très souvent indécis quand à l'utilisation ou non d'un point précis de la compétence.
% Valider étape après étape -- effectuer une mise en oeuvre intelligente (différente du Master 1 ou de la licence) 
% Intérêt : faire le lien entre l'automatique linéaire et l'implémentation 
% Compétences


En ouverture de ces travaux, nous remarquons qu'il est important d’appréhender de manière plus objective les compétences qui seront utilisé. Nous savons aussi maintenant à quel point il est essentiel de hiérarchiser la conception, en passant par chaque étape de validation qui ont été choisi avec précautions et toute connaissance du problème posé.
% Difficulté : choisir qu'est ce qui peut etre lié et ce qui n'est pas utile dans la construction de l'implémentation
% --> Lien entre systèmes linéaire/ non linéaires/ temps discret/ Microontroleur/
% Qu'est ce qui reste
% --> meilleure apréciations des compétences qui seront utilisé
% --> améliorer la hiérarchisation de la conception
% --> 