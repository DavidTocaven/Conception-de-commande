\chapter{Commande temps discret}

\section{Contraintes hardware} 
	Nous allons dans cette partie nous consacrer à une étude des contraintes matérielles. Ces contraintes seront à prendre en compte lorsque que nous passerons à la mise en œuvre.  
	\subsection{CAN / CNA}
		(protocole correction, temps conversion, échantillonnage bits )				
	\subsection{Micro contrôleur C167}
		(ordo,taches,validation TR,temps calcul, fréquence fonctionnement)
	\subsection{Conclusion}
	  	(contraintes tempo, squelette code correcteur)
\section{Discrétisation de la commande}
	\subsection{Fonctions de transferts}
		 (observateur + retour état = 2 ft)
	\subsection{Transformée en z}